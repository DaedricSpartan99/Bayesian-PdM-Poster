\headerbox{Subset simulation}{name=suspck,column=0,row=0,below=problem}{
Subset Simulation (SuS) \cite{sus} is suitable for the estimation of the probability of rare events, such as the failure probability in a SR framework. 
%allowing to computationally solve the Bayesian inverse problem.

\par\noindent
\\
The failure domain $F$ and it's associated probability $\mathbb{P}(F)$ are determined by a stochastic search procedure involving $r+1$ subsets of $\Dx$:

\begin{itemize}
\item {\smaller
$ F = F_{r+1} \subset F_r \subset F_{r-1} \subset \dots \subset F_1 \subset  F_0 = \Dx$
}
\item {\smaller $\mathbb{P}(F) = \prod\limits_{i=1}^{r+1} \mathbb{P}(F_i|F_{i-1}) =: \left( \frac{1}{10}\right)^r \mathbb{P}(F_{r+1}|F_r)$
}
\end{itemize}

%\par\noindent
%A limit state function is 
%Samples belonging to $F_r$ are determined such that at least $10\%$ of them are also BuS posterior samples, as they lie on the BuS failure domain $F$. Their distribution is a truncation of the prior:
Samples from each subset are generated by a MCMC routine which targets the distribution:
{\small$$ \pi(p, \vec{x}|F_i) \propto \mathbbm{1}_{F_i}(p, \vec{x}) \pi(\vec{x}), \quad 
i = 0,\dots,r+1$$}

The algorithm stops ($i = r$) when $h(p, \vec{x}) \le 0$ for at least $10\%$ of the samples.
$\mathbb{P}(F_{r+1}|F_r)$ is Monte Carlo estimated with $F_r$ samples.
}
