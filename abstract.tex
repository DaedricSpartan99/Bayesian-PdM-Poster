\headerbox{Abstract}{name=abstract,column=0,row=0}{

%The computation of likelihood function evaluation can take minutes or even hours, it's then fundamental to limit the number of likelihood calls.
%Active Learning Reliability (ARL) \cite{arl, bbus} is a framework exploiting surrogate model techniques and uncertainty quantification which allows to gain information around the limit state surface while limiting at the same time the number of forward model evaluations.
In this master thesis we explore and extend a recently proposed method to perform \textit{Bayesian inversion}, focusing on the reconstruction of the \textit{likelihood function} by combining \textit{active learning} \cite{arl} and \textit{Polynomial chaos Kriging} (PCK) \cite{pck}.
The basic idea is exploiting the recently proposed framework of \textit{Bayesian Updating with Structural Reliability} (BUS) \cite{bus}, which allows one to express the general \textit{Bayesian inference} formulation into a \textit{Structural Reliability} (SR) problem and rare event estimation.
\par\noindent
The \textit{active learning} approach allows to fix a limited budget to the \textit{forward model} evaluations for the PCK reconstruction, thus furnishing a cheap evaluator of the likelihood function in the most important regions.
%More specifically, a Polynomial Chaos Kriging (PCK) \cite{pck}  surrogate is adopted for the log-likelihood function and SuS is used to computationally solve the Bayesian inverse problem. 
}